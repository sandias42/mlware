\documentclass[11pt]{article}

\usepackage{common}
\title{Practical 2: Malware Classification}
\author{Ethan Alley, Grigory Khimulya, Walter Martin \\ alley@college.harvard.edu, khimulya@college.harvard.edu, wmartin@college.harvard.edu }
\begin{document}


\maketitle{}

\section{Technical Approach}

We explored the problem of malware classification from several different angles:

\begin{itemize}
	\item Running through a variety of models quickly to test which show the most immediate promise
	\item Feature engineering
	\item More focused tuning of hyperparameters for neural nets
\end{itemize}

How did you tackle the problem? Credit will be given for:

  \begin{itemize}
  \item Diving deeply into a method (rather than just trying
    off-the-shelf tools with default settings). This can mean 
    providing mathematical descriptions or pseudo-code.
  \item Making tuning and configuration decisions using thoughtful experimentation.  
    This can mean carefully describing features added or hyperparameters tuned.
  \item Exploring several methods. This can contrasting two approaches
    or perhaps going beyond those we discussed in class.
  \end{itemize}

  \noindent Thoughtfully iterating on approaches is key.
  If you used existing packages or referred to papers or blogs for ideas,
  you should cite these in your report. 

  \begin{table}
    \centering
    \begin{tabular}{@{}lll@{}}
      \toprule
      &\multicolumn{2}{c}{Mention Features  } \\
      & Feature & Value Set\\
      \midrule
      & Mention Head & $\mcV$ \\
      & Mention First Word & $\mcV$ \\
      & Mention Last Word & $\mcV$ \\
      & Word Preceding Mention & $\mcV$ \\
      & Word Following Mention & $\mcV$\\
      & \# Words in Mention & $\{1, 2, \ldots \}$ \\
      & Mention Type & $\mathcal{T}$ \\
      \bottomrule
      
    \end{tabular}
    \caption{Feature lists are a good way of illustrating problem specific tuning.}
  \end{table}



\section{Results}
This section should report on the following questions: 

\begin{itemize}
\item Did you create and submit a set of
  predictions? 
  

\item  Did your methods give reasonable performance?  
\end{itemize}

\noindent You must have \textit{at least one plot or table}
that details the performances of different methods tried. 
Credit will be given for quantitatively reporting (with clearly
labeled and captioned figures and/or tables) on the performance of the
methods you tried compared to your baselines.



\begin{table}
\centering
\begin{tabular}{llr}
 \toprule
 Model &  & Acc. \\
 \midrule
 \textsc{Baseline 1} & & 0.45\\
 \textsc{Baseline 2} & & 2.59 \\
 \textsc{Model 1} & & 10.59  \\
 \textsc{Model 2} & &13.42 \\
 \textsc{Model 3} & & 7.49\\
 \bottomrule
\end{tabular}
\caption{\label{tab:results} Result tables can compactly illustrate absolute performance, but a plot may be more effective at illustrating a trend.}
\end{table}




\section{Discussion} 


End your report by discussing the thought process behind your
analysis. This section does not need to be as technical as the others 
but should summarize why you took the approach that your did. Credit will be given for:

  \begin{itemize}
  \item Explaining the your reasoning for why you seqentially chose to
    try the approaches you did (i.e. what was it about your initial
    approach that made you try the next change?).  
  \item Explaining the results.  Did the adaptations you tried improve
    the results?  Why or why not?  Did you do additional tests to
    determine if your reasoning was correct?  
  \end{itemize}
 

\end{document}

